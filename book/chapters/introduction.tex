\setchapterpreamble[u]{\margintoc}
\chapter{Introduction}

Computer science is ths study of what we can compute. Computer scientists try to
answer questions like, are there unavoidable limits on the programs we can
write? (The answer is \emph{yes}. Some programs are impossible to write no
matter what.) And are there unavoidable limits on how slow some programs have to
run? (The answer again is yes. We'll see an example of a performance limit later
on.) Software engineering is concerned not with whether it is possible to build
something---they take the fact that their project can be built for granted---but
rather how can the project be built \emph{cheaply}. Software engineers are
concerned with how much the project costs, how much time the software takes to
run, how much memory it consumes---and, most importantly, whether the whole
thing can be shipped to the customer on schedule.

Sucessful programmers need to use a mix of tricks from computer science and
software engineering. Generally the two fields grow from one another. Engineers
sometiems stumble onto a new, unsolved problem during the course of working on
their projects. In the process of completing the project, new computer science
is made. Researchers cook up new algorithms to solve outstanding problems in the
field. In the process they invent new programming techniques. Those new
techniques make it back to the engineers, who then apply them in new projects.
And so the cycle goes on and on. What computer scientists and software engineers
do aren't that different. They approach similar problems with different
perspectives. To be a sucessful programmer, you will sometimes want to think
like a computer scientist and sometimes like a software engineer. Fortunately,
both perspectives come from the same underlying idea of computation, which can
be drawn very neatly in a diagram---the \emph{fundamental diagram}.

\section{The Fundamental Diagram}
The \textbf{Fundamental Diagram} is at the heart of everything software
engineers and computer scientists do.\sidenote{Full disclaimer: I might be the
only person who calls this the Fundamental Diagram. But everyone should. Like
reading, this diagram is fundamental.} As a programmer, you'll get to know this
picture very well. It shows up in just about every concept you'll ever
encounter---from the very big, to the obnoxiously small.

\begin{figure}[h]
  \begin{scaletikzpicturetowidth}{\textwidth}
  \begin{tikzpicture}[scale=\tikzscale, node distance=3cm, color=cyan, font=\sffamily\small]

    \node[draw, thick, fill=cyan!20, minimum size=2em, inner sep=1em] (transformer) {Transformer};
    \node (inputs) [left of=transformer] {Inputs};
    \node (outputs) [right of=transformer] {Outputs};

    \draw[->, thick, shorten >= 1em, shorten <= 0.25em] (inputs) -- (transformer);
    \draw[->, thick, shorten >= 0.25em, shorten <= 0.5em] (transformer) -- (outputs);

  \end{tikzpicture}
\end{scaletikzpicturetowidth}

\caption{\label{fig:intro-fundemental-diagram} The Fundamental Diagram of
computer science and software engineering in all its glory. It describes just
about everything a programmer could possibly do.}
\end{figure}

Fortunately, the diagram doesn't have many parts. There are just three: the
inputs, the outputs, and the transformer. The transformer takes in \emph{inputs}
and \emph{transforms} them into \emph{outputs}. That's it. As a programmer, you
will write software that turns specific inputs into particular outputs to
accomplsh a given task. If that description sounds vague, that's because it is.
Let's look at a few examples to see the Fundamental Diagram in action to get a
better idea of what's going on.

\paragraph{Example (Question and Answer)} A lot of human interaction follows the
input-transformer-output pattern. Questions and answers can be modeled by the
fundamental diagram pretty easily. For example, ``What's your favorite
color?''\sidenote{The answer is orange. Well, it is if I am the transformer
being asked. You might come up with a different but correct answer. Different
transformers can give different answers to the same question.} But what is the
transformer in this case? The transformer is the person who gives the answer to
the question; she listens to a question asked by somebody else, processes what
it means, and then produces an answer. For historical reasons---think ancient
Greek history---sometimes the role of the transformer in a Q\&A session is
called an \emph{oracle}.

\begin{figure}[h]
  \begin{scaletikzpicturetowidth}{\textwidth}
  \begin{tikzpicture}[scale=\tikzscale, node distance=3cm, color=cyan, font=\sffamily\small]

    \node (inputs) {Question};
    \node (question) [below=0em of inputs] {What is your favorite color?};

    \node[draw, thick, fill=cyan!20, minimum size=2em, inner sep=1em] (transformer) [right of=inputs] {Oracle};
    \draw[->, thick, shorten >= 1em, shorten <= 0.25em] (inputs) -- (transformer);

    \node (outputs) [right of=transformer] {Answer};
    \node (answer) [below=0em of outputs] {Orange.};
    \draw[->, thick, shorten >= 0.25em, shorten <= 0.5em] (transformer) -- (outputs);

  \end{tikzpicture}
\end{scaletikzpicturetowidth}

\caption{\label{fig:intro-example-question-answer} Fundamental diagram for a
question, oracle, and answer.}
\end{figure}

\paragraph{Example (Request and Response)}The same can be said of slightly
different kind of speech act called a \emph{request}. People make requests all
the time. ``Can you close the door, please?'' ``May I have the lobster sandwich
and a raspberry lime rickey?'' ``That'll be \$49.99, please.'' Each of these
statements intend to turn simple words into actions that produce some physical
effect---producing closed door, a lobster sandwich and a raspberry lime rickey,
or money totaling \$49.99.

\begin{question}
  \label{question:intro-transformers}
  Identify the inputs, outputs, and transformers for the requests above.
\end{question}

\begin{question}
  Draw the fundamental diagrams for each of the requests above.
\end{question}

On the web, requests are ubiquitous. When you open a web browser and type in the
address of a web page, the web browser makes a formal request to the machine
located at that address. \sidenote{In the same way individual people have street
addresses---like Sirius Black, who resided at 12 Grimmauld Place,
London---computers have internet addresses, too. These computers are called
\emph{web servers}. Like the computer which resides at the address
\texttt{www.dafk.net}.\vskip 5pt Individual web pages reside on web servers. For
example, the webapge \texttt{what} resides on the computer with the address
\texttt{www.darfk.net}. Its full address is \texttt{www.darfk.net/what}.} The
terminology for the actors in a web request comes from the restaurant industry.
At restaurants, clients make requests to people called servers, who fulfill
their clients' requests (usually with the help of many other people who are not
client-facing).

This pattern of request and response is exactly how the world wide web works.
The piece of software that initiates the request is called the \emph{client}.
Because this is the software that human users can see, people sometimes say
client technology is \emph{front end} technology. The software that receives,
processes, and sends back a response to the client is called the \emph{server}.
The server is not something that human users interact with directly. We don't
get to see what the server is doing, only the response that it produced. This is
a lot like how clients at a restaurant don't get to interact with the kitchen
directly. Instead they get to experience the finished product---but they don't
get know how the sausage is made, so to speak. Because the server's work to
transform a request into a full web page response is opaque to the user, we say
that servers a \emph{back end} technology.

\begin{figure}[h]
  \begin{scaletikzpicturetowidth}{\textwidth}
  \begin{tikzpicture}[scale=\tikzscale, node distance=3cm, color=cyan, font=\sffamily\small]

    \node (inputs) {Request};
    \node[minimum size=2em, inner sep=1em] (question) [below=0em of inputs] {\texttt{www.dafk.com/what}};

    \node[draw, thick, fill=cyan!20, minimum size=2em, inner sep=1em] (transformer) [right of=inputs] {Server};
    \node (server) [below=0em of transformer] {\texttt{www.dafk.com}};
    \draw[->, thick, shorten >= 1em, shorten <= 0.25em] (inputs) -- (transformer);

    \node[minimum size=2em, inner sep=1em] (outputs) [right of=transformer] {Response};
    \node (response) [below=0em of outputs] {Contents of \texttt{what}};
    \draw[->, thick, shorten >= 0.25em, shorten <= 0.5em] (transformer) -- (outputs);

  \end{tikzpicture}
\end{scaletikzpicturetowidth}

\caption{\label{fig:intro-example-request-response} Fundamental diagram for a
request, server, and response.}
\end{figure}

The client-server architecture, like the fundamental diagram, is deceptively
simple and yet it powers the whole of the internet.\sidenote{Software engineers
call the organization of data flows an \emph{architecture}. In this case, the
flow of information is the request from the client to the server and the
response from server to client.} Web programming is a specialized collection of
technologies, many of which we'll encounter in this series. When you're
bushwacking the jargon and buzzwords of the web technology jungle, stop and
remember Figure \ref{fig:intro-example-request-response}: just two arrows and a
box in the middle. Almost all of the latest and greatest technologies are just
fancy ways to request some piece of data from a server. Always keep this picture
in mind.

\paragraph{Example (Making Ice Cream)} Recipes are hands-down the tastiest kinds
of transformers.Recipes explain how to transform ingredients into prepared
dishes. Here's one I like because it's quick, easy, and the result is something
near and dear to my heart: ice cream.

\begin{marginfigure}
  \begin{scaletikzpicturetowidth}{\textwidth}
  \begin{tikzpicture}[scale=\tikzscale, color=cyan, font=\sffamily\footnotesize]
    \node (input) at (0,0) {};
    \node[draw, thick, fill=cyan!20, minimum size=1em, inner sep=0.5em] (transformer) at (1,0) {Recipe};
    \node (output) at (2,0) {};

  % https://thenounproject.com/search/?q=milk&i=1577815
  % Milk by Wawan Hermawan from the Noun Project
    % \draw[fill=cyan, rotate=180, scale=0.15] (0,0) svg "M50.5283203,62.3232422c-0.4343262-2.0397949-2.357605-3.3642578-4.3845215-3.1366577   c0.6894531-0.8180542,1.1194458-1.965271,0.9138184-3.4907837c-0.1113281-0.8208008-0.8583984-1.3964844-1.6865234-1.2866211   c-0.8212891,0.1108398-1.3964844,0.8657227-1.2861328,1.6870117c0.119812,0.8925171-0.2557983,1.3293457-0.717041,1.5389404   c-0.9266357-0.5111084-1.9899902-0.8035889-3.1208496-0.8035889c-1.1310425,0-2.194519,0.2926025-3.1212769,0.803833   c-0.4609375-0.2095337-0.8362427-0.6469116-0.7166138-1.5391846c0.1103516-0.8212891-0.4648438-1.5761719-1.2861328-1.6870117   c-0.8300781-0.109375-1.5761719,0.4658203-1.6865234,1.2866211c-0.2055054,1.5245361,0.2238159,2.6713257,0.9122925,3.4892578   c-0.9240723-0.1054688-1.8487549,0.0901489-2.638855,0.6030273c-0.9013672,0.5844727-1.5205078,1.4848633-1.7441406,2.5351563   c0,0,0,0,0,0.0004883c-0.0957031,0.4482422-0.109375,0.9208984-0.0390625,1.4038086   c0.0625,0.4350586,0.3125,0.8208008,0.6845703,1.0556641c0.4121094,0.2597656,0.8486328,0.4404297,1.2978516,0.5361328   c0.28125,0.0595703,0.5644531,0.0893555,0.8457031,0.0893555c0.4030762,0,0.8010864-0.0646362,1.1846924-0.1845093   c0.2687988,1.8319702,0.8359985,4.2652588,1.6209717,6.3978882c0.2167969,0.5898438,0.7783203,0.9819336,1.4072266,0.9819336   h6.5576172c0.6289063,0,1.1904297-0.3920898,1.4072266-0.9819336c0.7855225-2.133606,1.3529053-4.5665894,1.6218262-6.3979492   c0.383606,0.119873,0.7816772,0.1845703,1.1848145,0.1845703c0.28125,0,0.5644531-0.0297852,0.8457031-0.0893555   c0.4462891-0.0952148,0.8818359-0.2753906,1.2958984-0.534668c0.3730469-0.234375,0.6240234-0.6210938,0.6865234-1.0571289   c0.0703125-0.4829102,0.0566406-0.9555664-0.0390625-1.4038086C50.5283203,62.3232422,50.5283203,62.3232422,50.5283203,62.3232422   z M42.4560547,69.6040039h-4.4189453c-0.8242188-2.5600586-1.2744141-5.2426758-1.2744141-6.2880859   c0-1.9213867,1.5634766-3.484375,3.484375-3.484375s3.484375,1.5629883,3.484375,3.484375   C43.7314453,64.3603516,43.2802734,67.0424805,42.4560547,69.6040039z";

    \draw[fill=cyan, rotate=180, shift={(-0.15,-0.22)}] svg[scale=0.13pt] "M24.7421875,96h27.2958984h3.7138672H74.5c0.828125,0,1.5-0.6713867,1.5-1.5v-62   c0-0.0164185-0.0042725-0.0316772-0.0048218-0.0479736c-0.001709-0.053772-0.008667-0.105957-0.0159302-0.1582031   c-0.0065918-0.0474243-0.0126953-0.0948486-0.0239258-0.1419678c-0.0107422-0.0448608-0.026001-0.0877075-0.0406494-0.1308594   c-0.0175171-0.0516357-0.0355225-0.1032104-0.059082-0.1536865c-0.0066528-0.0143433-0.0095825-0.0293579-0.0167236-0.0435791   l-8.9316406-17.6810303V5.5c0-0.8286133-0.671875-1.5-1.5-1.5H34.5927734c-0.828125,0-1.5,0.6713867-1.5,1.5v8.267334   c-0.0049438,0.0088501-0.0127563,0.0151978-0.0175781,0.0241699l-9.6494141,18   c-0.0020752,0.0038452-0.0025024,0.0081177-0.0045166,0.0119629C23.31073,32.0124512,23.2421875,32.2470703,23.2421875,32.5v62   C23.2421875,95.3286133,23.9140625,96,24.7421875,96z M52.0380859,93H26.2421875V34h25.7958984h2.2138672v59H52.0380859z M73,93   H57.2519531V34H73V93z M72.0617676,31H58.2639771l7.1038208-13.2514038L72.0617676,31z M36.0927734,7h27.8144531v6H36.0927734V7z    M35.2949219,16h27.6064453l-8.0410156,15h-2.8222656H27.2539063L35.2949219,16z";

    % https://thenounproject.com/search/?q=ice%20cream%20cone&i=1422432
    % Ice Cream by Landan Lloyd from the Noun Project
    \draw[fill=cyan, rotate=180, shift={(-2.40,-0.35)}] svg[scale=0.2pt] "M64.71,53a5.12,5.12,0,0,0-.78-2.71,13.46,13.46,0,0,0-.23-5.49,5.71,5.71,0,0,0,.45-5.44c0-.06,0-.11,0-.16a12.7,12.7,0,0,0-3-9.58A5,5,0,0,0,61.36,28a5.15,5.15,0,0,0-1-2.69,10.74,10.74,0,0,0-10.61-8.94l-.67,0A10.74,10.74,0,0,0,39,27.52a5.45,5.45,0,0,0-.33,2.27,5.34,5.34,0,0,0,.52,2,12,12,0,0,0-1.31,3.86,5.87,5.87,0,0,0-1,2.71,5.71,5.71,0,0,0,.72,3.47,13.88,13.88,0,0,0-1.47,6.25v.39a5.79,5.79,0,0,0,2,7.92l7.66,23.94a4.5,4.5,0,0,0,8.64,0l7.29-22.67A5.13,5.13,0,0,0,64.71,53Zm-3.6-4.9a11.06,11.06,0,0,1-.45,3.07A2.12,2.12,0,0,1,61.71,53a2,2,0,0,1,0,.25h0a3.61,3.61,0,0,1-.08.38h0a1.43,1.43,0,0,1-.14.34v0a1.81,1.81,0,0,1-.19.29l0,.05-.21.22-.09.07a1.66,1.66,0,0,1-.21.16l-.15.08-.19.1-.23.07-.15,0a1.84,1.84,0,0,1-.41,0l-.26,0h0a2.11,2.11,0,0,1-1.83-1.79h0a2.77,2.77,0,0,1-.71-.1,2,2,0,0,1-1.84,1.21,2,2,0,0,1-1.38-.56,2.77,2.77,0,0,1-4.18-.19,1.16,1.16,0,0,1-.9.43,1.18,1.18,0,0,1-1-.52,3.08,3.08,0,0,1-1.57.43,3.16,3.16,0,0,1-1.19-.24,1.72,1.72,0,0,1-3.19.47,3.13,3.13,0,0,1-.44,0,3,3,0,0,1-.52,0h0a3,3,0,0,1-.72-.24l-.08,0a2.6,2.6,0,0,1-.59-.41h0a3,3,0,0,1-.8-1.34h0a2.84,2.84,0,0,1-.1-.72,2.77,2.77,0,0,1,.91-2.06,9.5,9.5,0,0,1-.09-1.31,11.05,11.05,0,0,1,.75-4A5.59,5.59,0,0,0,42,44.8a4.54,4.54,0,0,0,.63,0,5.76,5.76,0,0,0,2.4-.53,5.81,5.81,0,0,0,1.37.32l.66,0a4,4,0,0,0,.49,0,5.77,5.77,0,0,0,3.83,2.06c.21,0,.43,0,.64,0a5.68,5.68,0,0,0,2.37-.51l.51.09.45,0a5.8,5.8,0,0,0,2.85,1.13l.63,0a5.81,5.81,0,0,0,2.21-.44A9.28,9.28,0,0,1,61.11,48.11ZM61.62,42a2.39,2.39,0,0,1-.14.6l-.06.16a2.32,2.32,0,0,1-.22.41l-.07.12a2.89,2.89,0,0,1-.91.83h0a2.82,2.82,0,0,1-1.38.37h-.31A2.84,2.84,0,0,1,56.37,43a1.6,1.6,0,0,1-1,.31h-.18a1.62,1.62,0,0,1-1.08-.57,2.79,2.79,0,0,1-2.12,1h-.31a2.8,2.8,0,0,1-2.48-3,3,3,0,0,1-2.17.92h-.33a3.07,3.07,0,0,1-1.86-.93,2.84,2.84,0,0,1-2.22,1.12h0l-.3,0a2.93,2.93,0,0,1-.94-.28h0l-.31-.18-.07,0a3,3,0,0,1-.76-.8v0A2.78,2.78,0,0,1,39.84,39a2.43,2.43,0,0,1,0-.27,2.82,2.82,0,0,1,.93-1.78c0-.09,0-.18,0-.27a9.45,9.45,0,0,1,.59-2.45,5.55,5.55,0,0,0,2.85.8h.34a5.42,5.42,0,0,0,2.3-.65,6.33,6.33,0,0,0,2,.32h.38a6.34,6.34,0,0,0,3.49-1.32h.34a5.36,5.36,0,0,0,1.12-.19,5.18,5.18,0,0,0,2,.41h.32a5.09,5.09,0,0,0,3-1.21,9.61,9.61,0,0,1,1.7,6.55c0,.36-.11.7-.18,1A2.78,2.78,0,0,1,61.62,42ZM42,27.56a7.71,7.71,0,0,1,7.28-8.13l.49,0a7.72,7.72,0,0,1,7.74,7.2s0,0,0,0a2.3,2.3,0,0,1,.52.6h0a3.55,3.55,0,0,1,.21.44s0,0,0,0a2.28,2.28,0,0,1,.1.52,2.32,2.32,0,0,1,0,.53l0,.13a2.7,2.7,0,0,1-.14.4.2.2,0,0,0,0,.08,2.21,2.21,0,0,1-.72.8h0a2.09,2.09,0,0,1-1.08.37h-.13a2.2,2.2,0,0,1-1.76-.9,2.4,2.4,0,0,1-1.54.68h-.14a2.5,2.5,0,0,1-1.06-.25,3.37,3.37,0,0,1-2.63,1.57h-.2a3.33,3.33,0,0,1-2.38-1A2.53,2.53,0,0,1,44.39,32h-.16a2.57,2.57,0,0,1-1.29-.35h0a2.63,2.63,0,0,1-.85-.81L42,30.74a3.11,3.11,0,0,1-.2-.41.78.78,0,0,0,0-.14,2.3,2.3,0,0,1-.1-1,2.14,2.14,0,0,1,.06-.35,2.66,2.66,0,0,1,.35-.83C42.07,27.88,42,27.73,42,27.56ZM51.5,79.49a1.51,1.51,0,0,1-2.9,0L41.71,57.93a4.63,4.63,0,0,0,1.34.21,4.72,4.72,0,0,0,3-1.12,5.49,5.49,0,0,0,1.07-.13,4.13,4.13,0,0,0,1.3.22l.42,0a5.81,5.81,0,0,0,4.87.2,5.06,5.06,0,0,0,1.17.14,5.18,5.18,0,0,0,1.67-.28,5.2,5.2,0,0,0,1.82.83Z";

    \draw[->, thick, shorten >= 0.25em, shorten <= 0.25em] (input) -- (transformer);
    \draw[->, thick, shorten >= 0.25em, shorten <= 0.25em] (transformer) -- (output);


  \end{tikzpicture}
\end{scaletikzpicturetowidth}

\caption{\label{fig:intro-example-ice-cream} Total preparation time is about 15
minutes plus a 20 minutes soft freeze in the ice cream maker followed by an
overnight hard freeze in the freezer.\vskip 5pt Milk by Wawan Hermawan and ice
cream by Landan Lloyd from the Noun Project.}
\end{marginfigure}

Up until now, each transformer we've described takes exactly one input and
produces a single output. Here is our first example of a transformer that takes
multiple inputs to produce its output. It is not uncommon for a transformer to
take multiple inputs. There are very few recipes that require just a single
ingredient. So, too, programs often need to use combine multiple inputs to
generate a desired output.

Let's look at the process to transform a few ingredients into ice cream in some
detail. Below is a kind of step-by-step list of instructions that transform the
input ingredients milk, cream, sugar, and salt, into ice cream. Each step goes
on its own line. As a general rule, steps should be kept as simple as possible.
Complex steps should be broken up into multiple, simpler steps. This strategy
makes it easier for the person preparing the recipe to understand and execute.
The programs we write to transform inputs into outputs will have a very similar
character.

\suppresslinenumbers
\begin{lstlisting}[caption={\label{listing:intro-ice-cream} Ice cream
transformer. It's as short as it is sweet.}, escapeinside=$$, firstnumber=0]
$\textbf{Inputs:}$ 1 c. whole milk
        2 c. heavy cream
        2/3 c. sugar
        1/2 tsp. Kosher salt

$\textbf{Output:}$ Sweet cream ice cream
$\reactivatelinenumbers$
Place the milk, heavy cream, sugar, and salt in a bowl.
Mix until the sugar has completely dissolved.
Pour the ice cream base into an ice cream maker.
Run the machine until the base mixture firms up.
Transfer the cold cream into a sealed container.
Chill the container in a freezer overnight.
\end{lstlisting}

Did you notice that our great big ice cream transformer is made up of three
smaller input-output transformers? The first one decribed on lines 1--2. It
turns the input ingredients into a liquid base mixture. The next one on lines
3--4 takes as input the base mixture and outputs a cold cream. While cold cream
is delicious, it melts extremely rapidly. And so the recipe finishes with the
transformer on lines 5--6, which takes as input the cold cream and freezes it
until to produce bona fide ice cream.

\begin{figure*}[h]
  \begin{scaletikzpicturetowidth}{\textwidth}
  \begin{tikzpicture}[scale=\tikzscale, color=cyan, node distance=3cm, font=\sffamily\small]

    % liquid base transformer
    \node (ingredients) {Ingredients};
    \node[draw, thick, fill=cyan!20, minimum size=2em, inner sep=1em] (mixer) [right of=ingredients] {Combine};
    \draw[->, thick, shorten >= 1em, shorten <= 0.25em] (ingredients) -- (mixer);

    \node (base) [right of=mixer] {Liquid Base};
    \draw[->, thick, shorten >= 0.25em, shorten <= 0.25em] (mixer) -- (base);

    % cold cream transformer
    \node[draw, thick, fill=cyan!20, minimum size=2em, inner sep=1em] (maker) [below=1cm of base] {Soft Freeze};
    \draw[->, thick, shorten >= 0.25em, shorten <= 0.5em] (base) -- (maker);

    \node (cold cream) [right of=maker] {Cold Cream};
    \draw[->, thick, shorten >= 0.25em, shorten <= 0.5em] (maker) -- (cold cream);

    % ice cream transformer
    \node[draw, thick, fill=cyan!20, minimum size=2em, inner sep=1em] (freezer) [below=1cm of cold cream] {Hard Freeze};
    \draw[->, thick, shorten >= 0.25em, shorten <= 0.5em] (cold cream) -- (freezer);

    \node (ice cream) [right of=freezer] {Ice Cream};
    \draw[->, thick, shorten >= 0.25em, shorten <= 0.5em] (freezer) -- (ice cream);
  \end{tikzpicture}
\end{scaletikzpicturetowidth}

\caption{\label{fig:intro-example-ice-cream-pipeline} The steps inside
transformers are often simpler transformers. The transfromation that produces
ice cream from ingredients is made up of three successive transformers. The
output of one transformer is the input to the next transformer.}
\end{figure*}

Figure \ref{fig:intro-example-ice-cream-pipeline} illustrates that old saying,
``One transformer's output is another transformer's input.'' And that is the
real power of the fundamental diagram. You can chain copies of it together to
produce more and more complicated flows of inputs and outputs. In this case, we
chained together three transformers in the simplest way possible, in a line, one
after the other. The \emph{pipeline} pattern is very common in software
architecture. In fact, many of the applications in ``big data'' reduce to
pipelines of transformers, no different conceptually to the ice cream pipeline
above. When we work with command line tools, we will meet the pipe operator that
feeds the output of one program as the input to another.\sidenote{The pipe
operator notated by the symbol `\texttt{|}`. It's that shift character on the
backslash key. Now go lay some pipe!} It acts like a pipe for the information to
flow through. Hence the name \emph{pipe}.

\begin{question}
In the style of Listing \ref{listing:intro-ice-cream}, write the inputs,
outputs, and steps to create a grilled cheese sandwich.
\end{question}

\begin{question}
  Draw a fundamental diagram for your grilled cheese listing.
\end{question}

\begin{question}
Come up with three more examples that can be modeled by the fundamental diagram.\end{question}

\section{Programs}
You will use a text editor to write your programs, because programs are a kind
of written text. You can think of them as a sort of genre. Programs follow the
rules and conventions of their genre---the same as a sales brochure or novel do.
And like any other work of literature, can be judged on any number of
dimensions---readability, structure, flow, rhythm---so, too, can the text (code)
of an individual program be judged sloppy or sublime. There are a lot of strong
opinions about how to organize code. And religious wars about what a good
program looks like break out all the time. And just as good writing takes lots
of practice and revision, so does writing a good program.

Writing a program is an iterative process, with many drafts, edits, and
restructuring. The practice of editing a program is so important that it has
been given a special name, \emph{refactoring}. Refactoring is the process of
restructuring code in a program to improve it somehow. Often the aim of
refactoring is understandability. Remember, programs are for human-readers, not
computers. \marginnote{Other programmers include \emph{you} in the future. You
think you'll remember what you were thinking when you wrote that obtuse line of
code, but I promise you, you will not.} So you should go to great pains to make
your code legible and digestable to other programmers. Other goals include
troubleshootability, extensibility, and performance.

Refactoring isn't separate from programming. It's part of writing, drafting, and
improving every program you write. Refactoring projects can be as small as
changing a variable name, or as large as a full-scale project rewrite. Usually
they range somewhere in between. Refactoring is an essential skill. We will
refactor many of the programs we write in this book because we refactor many of
the programs we write on the job.

Most genres contain subgenres. Within the novel you find mystery novels, romance
novels, travel novels, and other that cross and combine subgenres. Programs
constitute a large genre. Quite naturally, the field has split in many kinds of
programming: systems programming, graphics programming, web application
programming, game programming, scientific programming, etc. All of kinds of
programming, however, rely on foundational concepts and common techniques.

Genre transcends language. Novels and sales brochures can be written in any
language. And the same general rules apply whether written in English or
Swedish. Programs can be written in a variety programming languages. Most
programming languages are designed for human readers and writers. The machines,
which execute the instructions written in code, however, speak their own
\emph{machine languages}. And the contents of the text files that programmers
produce need to be converted into directives that can be carried out reliably
and directly on hardware. There are two special kinds of programs that perform
the translation from a source programming language to machine language:
compilers and interpreters. The difference is that compilers take in a full
program and convert it all at once. Interpreters take in one statement of
programming at a time and convert on the fly. (Much the same difference between
translating a book---the compiler---and what someone is saying in real-time---an
interpreter.)

\section{Compilers} One of the most important transformer in programming is type
of program called a \emph{compiler}. Compilers translate programs from one
language into another---specifically, from a \emph{source language} that humans
can read, write, and understand to a \emph{target language} that computers can
interpret and execute.

\begin{figure}[h]
  \begin{scaletikzpicturetowidth}{\textwidth}
  \begin{tikzpicture}[scale=\tikzscale, node distance=3cm, color=cyan, font=\sffamily\small]

    \node (inputs) {Program Instructions};
    \node (question) [below=0em of inputs] {JavaScript/Python/etc.};

    \node[draw, thick, fill=cyan!20, minimum size=2em, inner sep=1em] (transformer) [right of=inputs] {Compiler};
    \draw[->, thick, shorten >= 1em, shorten <= 0.25em] (inputs) -- (transformer);

    \node (outputs) [right of=transformer] {Bytecode};
    \draw[->, thick, shorten >= 0.25em, shorten <= 0.5em] (transformer) -- (outputs);

  \end{tikzpicture}
\end{scaletikzpicturetowidth}

\caption{\label{fig:intro-example-compiler} Fundamental diagram for source code
compilation into machine-readable bytecode.}
\end{figure}

It may seem surprising that the code you write is \emph{not} for the computer
but for people---like yourself! The source code that you write needs to be
converted into machine-friendly instructions that can be executed by hardware.
Compilers make programming accessible to people who do not hold PhDs in
mathematics, and they are one of the great achievements in the history of
computing. Without compilers we would need to know absolutely everything about
the hardware your program runs on---how its memory is structured, how many
registers it has---and to manually keep track of and manipulate the location of
your data in memory. You can program in a \emph{low level language} like
Assembly if you want to get cozy with the bare metal. Since there are so many
more things to keep track, low-level language programs tend to be very long and
hard to follow. By comparison, \emph{high level languages} cannot be executed by
the machine directly and require an extra compilation step that generates the
low-level code for you. Generally, this extra step is worth it. Programs written
in high level languages are shorter and easier to follow than their low level
counterparts.

As an example, compare two implementations of the classic program \textsf{Hello,
World}. The first one is written in NodeJS, a high level language.

\begin{lstlisting}[caption={\label{listing:intro-hello-world-nodejs} Hello,
World! in NodeJS.}, escapeinside=$$, firstnumber=1]
// hello.js
"use strict";

console.log("Hello, World!");
\end{lstlisting}

The second implementation is written in Assembly, a low level language that
mimics the architecture of the hardware that the program runs on.

Be very thankful that you have a friendly compiler that can translate your
directions into machine-readable instructions.

\textbf{TODO: Hello World in Assembly}
https://tldp.org/HOWTO/Assembly-HOWTO/hello.html Say that assembly maps 1:1 to
machine architecture using an \emph{assembler}. Often source code gets compiled
to assembly instructions then assembled into machine instructrions. One
assembler per type of chip architecture. Why talk about this at all? The
important part is that software needs to run on hardware. At the end of the day,
there is a physical location to all of the information in a program. (What about
virtualization? Well, those virtualizations model the hardware, so they have to
model the physical location, too. And those virtualizations run on real
hardware. So even programs that simulate computers need to run on real
computers. The data has to live in a physical location at some point. It can't
be turtles \emph{all} the way down.)

Make a fundamental diagram of source code $\to$ compiler $\to$ assembly code
$\to$ assembler $\to$ machine instructions.

\textbf{TODO: Grace Hopper.
Fran Allen:
https://www.ibm.com/blogs/research/2020/08/remembering-frances-allen/}

\section{Interpreters}

Interpreters are an interactive version of compilers. Instead of writing all of
the code out at once and compiling it to a finish program, interpreters let you
type in instructions one at a time and run your program incrementally. They are
a playground, a sandbox to try things out.

[Insert fundamental diagram for interpreter]

Reference \textit{Structure and Interpretation of Computer Programs}.

JavaScript is an interpreted language. It is useful to try things out in an
interpreter while writing a program. We will use the NodeJS interpreter a lot to
develop our programs as we write them.

\textbf{TODO:} JavaScript is based on another language called ECMAScript, which
is specified by an standardization board called ECMA International. ECMA
approves standards documents, written in English, for new language features,
changes, and extensions. The standards body publishes new editions to the
standards documents of ECMAScript with some regularity. Once standards have been
approved, developers start implementing the features in a JavaScript
interpreter. (Although sometimes it works the other way around. Developers cook
up a feature that goes viral, and then the standards body adopts the feature as
an official additional to the language.)

There is no rule that says a JavaScript interpreter must implement the standard
fully---though, it's less useful if an implementation is not
standards-compliant. Each of the major web browsers, for example, implement
their own JavaScript engine. From time to time one browser will implement a
feature that others have not and vice versa. In this sense, each browser runs a
different version or flavor of JavaScript, because, as we've said before---a
programming language has to be implemented itself as a program. The Mozilla
Foundation keeps track of compliance to ECMAScript standards for each of the
major browsers on the MDN Web Docs
page\sidenote{https://developer.mozilla.org/en-US/docs/Web/JavaScript}.

JavaScript is not bound to web browsers, however. Due in large part to the
effort put into the open-source implementation Node.js\sidenote{nodejs.org},
JavaScipt can be used to general programming outside of the web. Node.js bills
itself as server-side JavaScript, and was created to make building fast,
scalable network applications for the web. Node.js distributions include extra
libraries that are useful for general-purpose programming, such as libraries to
manipulate the file system. In this book we will be using Node.js as our
reference implementation. While many of the exaples in this book will work in a
web browser developer's console, some of them will explicitly use features
available only in Node.js.


\section{Four Friends of Programming}

There are lots of things to learn about engineering. New technologies are built
every single day. And it is difficult to keep up. The shear number of things to
know can be overwhelming. But fear not! There actually aren't that many
different things you can do with a programming language. In fact, there are only
four.\marginnote{The fact that there are only four things you can do is
something of a theoretical marvel. It is related to \emph{Church-Turing thesis}
and lies as the heart of the foundations of computing. Lucky for us, it means we
need to learn only four concepts. Actually the fourth one, organization and
reuse, is merely a convenience for us humans.} Any new technology or technique
you run across is a cousin to one or more of the four friends you will make in
this chapter. The four things you can do in a programming language are:

\begin{enumerate}
  \item \textbf{Store data and retrieve it later to use.}
Your friend for the pillar of data storage and retrieval is a \emph{variable}.
Other instances of this pillar include files, databases, networked resources,
and web pages on the Internet. At their heart, however, they're all about
stashing some information somewhere and fetching it later.

  \item \textbf{Decide between alternatives.}
Your friend for the pillar of decision-making is the \textsf{if} statement.
Other examples of this pillar include associative arrays, pattern matching, and
\textsf{switch} statements. These technologies exist to add decision-making into
programs. Conditional elements put the ``sometimes'' into programs that would
otherwise be ``always'' based on context and circumstance.

  \item \textbf{Repeat things.}
The ability to repeat things quickly is what makes computers so useful. Your
friend for the pillar of repetition is the \textsf{for}~loop. Other examples of
this pillar include \textsf{while}~loops and \textsf{do-while}~loops, recursion,
and coroutines. At their core, however, the whole point is to repeat something.

  \item \textbf{Organize 1--3 into individual pieces.}
This last pillar isn't technically required, but it perhaps the most
useful.\marginnote{See, for example, the controversy about whether or not to put
curly braces on their own line or not:
https://en.wikipedia.org/wiki/Indentation\_style.} It's all about organizing our
programs into pieces that are easy for people to read, understand, expand,
troubleshoot, and fix. This pillar could also be called \emph{programming
style}. Your friend from this pillar is \emph{function}. Other members of the
pillar include libraries, modules, closures, classes, and design
patterns\marginnote{People write full books on programming style and they almost
always disagree on what ``best'' practices are.}. Since style is often a matter
of taste, this pillar is the most contentious and is the basis of many religious
wars.
\end{enumerate}

That's it! Master the four pillars of programming and you can confidently tackle
any project. That is not to say that these four skills are easy to master.
Mastery is hard work. But when you find yourself learning something new, ask
yourself, ``Which of my four friends is this new friend most like?'' Or if you
are stuck implementing a solution and aren't sure what to do, ask yourself,
``Which of my four friends could help me most right now?''

\textbf{TODO: Add questions about the four pillars/abilities.}

\textbf{TODO: Remark that programming languages are programs.} Compilers and
interpreters are programs themselves, and therefore need to be programmed in a
programming language.

If ther are only four things you can do in any programming language, why are
there so many? You need to be able to do 1--3 in any general programming
language. How do you do that and how you organize the code is all up to 4.
Programming languages fall into certain ways of organizing your thoughts---these
styles of thinking and orgaization are called paradigms. Most languages will be
multi-paradigm.

Examples Imperative: C, Algol, Fortan, NodeJS. Declarative: SQL, Metafont,
Prolog. Object based: Java. Stack-based: PostScript, Forth. Functional: Haskell.
Erlang. List-based: LISP, Scheme. Array-based: Matlab, Julia, Numpy.
Multi-paradigm: Python, Swift.\marginnote{There are as many paradigms as there
are ways of thinking. APL is insane-o. Here is a program that finds prime
numbers in APL. \texttt{PRIMES : (\textasciitilde
R$\in$R$\circ$.$\times$R)/R$\leftarrow$1$\downarrow$$\iota$R}}

Languages make some tasks easier, but at the cost of making other harder. Bob
Marley, ``Truth is everybody is going to hurt you: you just gotta find the ones
worth suffering for.'' Everything is a trade-off in design, including
programming language design. You can technically create a mobile app in Julia,
but it would be very, very, very hard. You can implement scientific computing in
NodeJS, but it would be very, very hard. Choose a good tool for the task at
hand.

Remember that programming languages are implemented by their compiler---the
compiler converts text files into machine executable code. But the compiler is
just a program, so it has to be written in a programming language. There is
source code for the compiler that a compiler compiles. The standard
implementation of Python is written in C, but Jython is Python implemented in
Java.

\section{Summary}
\textbf{TODO}
